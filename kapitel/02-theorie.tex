%
% Grundlagen
%
% @version 1.0
% @author wipatrick
% @created 22. November 2015
% @edited 

\chapter{Grundlagen}
\label{chap:grundlagen}

Ich bin ein \textbf{Paper} \citep[S. 234]{Chang2006}.\newline
Ich bin ein \textbf{Buch} \citep[S. 20f.]{Goll2014}. \newline
Ich bin eine \textbf{Buch-Section} \citep[S. 55ff.]{Bolanos2013} \newline
Ich bin ein \textbf{Inproceedings} \citep[]{Rodenberg2013} \newline
Ich bin eine \textbf{Internetquelle} \citep[]{Caserta2014}. \newline

Dies ist \acrshort{zb} eine \ref{tab:zsmmuster} mit dem Paket \texttt{\textbackslash usepackage\{threeparttable\}}.

\begin{table}[H]
\centering
\small
\renewcommand{\arraystretch}{1.3}
\begin{threeparttable}
\begin{tabularx}{\textwidth}{p{0.018\textwidth}p{0.25\textwidth}C{0.1915\textwidth}C{0.1915\textwidth}C{0.1915\textwidth}}
\toprule

\multicolumn{1}{c}{N\textdegree}		&
Characteristic						&
\multicolumn{3}{c}{Instances} \\

\cmidrule[0.4pt](r{0.25em}){1-1} 
\cmidrule[0.4pt](lr{0.25em}){2-2}
\cmidrule[0.4pt](l{0.25em}){3-5}
%\cmidrule[0.4pt](lr{0.25em}){4-4}
%\cmidrule[0.4pt](l{0.25em}){5-5}

1						&
Apache Project			&	
\ccol Storm	 	        &	
Spark			        & 	
Flink 					\\

2						&
Abstraction	            &	
\ccol Bla				&	
Bla					    &	
Bla 					\\

3						&
Bla			            &	
\ccol Bla		        &	
Bla				        &	
Bla 			        \\

4						&
Bla\tnote{1)}		    &
\multicolumn{2}{c}{\ccol Bla}   &
Bla 							\\

\bottomrule
\end{tabularx}
\begin{tablenotes}[]\footnotesize\singlespacing\setlength\labelsep{0pt}
\item[\textcolor{black!20}{\quadrat}] Der Fokus dieser Arbeit liegt auf ...
\item[1)] Die ist eine weitere Fußnote.
\end{tablenotes}
\end{threeparttable}
\caption[Tabellenunterschrift im Tabellenverzeichnis]{Tabellenunterschrift im Test.}
\label{tab:zsmmuster}
\end{table}

Dies ist eine \ref{fig:abb1}. Das Bild ist ein Platzhalter. Das macht aber nichts.

\begin{figure}[H]
\centering
\includegraphics[width=\textwidth]{example-image-a}
\caption[Abbildungsunterschrift im Abbildungsverzeichnis]{Abbildungsunterschrift im Test}
\label{fig:abb1}
\end{figure} 

Das ist Pseudocode \ref{fig:pseudocode}.
\IncMargin{1.5em}
\begin{algorithm}
\onehalfspacing
%\vspace*{-.9cm}
\begin{figure}[H]\small
  \SetKwInOut{Input}{Eingabe}
  \SetKwInOut{Output}{Ausgabe}
\Indm  
\Input{Testpunkt $p=(p_x, p_y)$, Polygon $P=(V, E)$, mit \\Menge Eckpunkte $V=\{v_1, v_2, \dots, v_n\}$ und \\Menge Kanten $E=\{(v_1,v_2), (v_2,v_3), \dots (v_{n-1}, v_n), (v_{n}, v_1)\}$}  \Output{Testpunkt $p$ innerhalb/außerhalb Polygon $P$}
\Indp
  \BlankLine  \BlankLine 
  \For{ alle Kanten $(v_i, v_j) \in E$}{ 
   \tcp{\textcolor{gray}{Prüfe, ob die Eckpunkte in unterschiedlichen Regionen liegen}}
   \If {$\Big((v_{i,y}<p_y)$ und $(v_{j,y}\ge p_y)\Big)$ oder $\Big((v_{j,y}<p_y)$ und $(v_{i,y}\ge p_y)\Big)$}{
  Berechne den Schnittpunkt $S =(S_x, S_y)$ zwischen Kante $e=(v_i, v_j)$ und \\
     dem Strahl $r=p_y$. Aus dem Strahlensatz folgt: $S_x = v_{i,x} + \Big( \frac{\Delta y}{\Delta Y} \cdot \Delta X\Big)$\;
  \tcp{\textcolor{gray}{Prüfe, ob der Schnittpunkt rechts des Testpunkts liegt}}
   \If {$S_x \ge p_x$}{Erhöhe den Zähler um $1$.}
}
}
\tcp{\textcolor{gray}{Nachdem alle Kanten durchlaufen wurden, tue Folgendes:}}
\eIf {Zähler ist ungerade, d.h. $2k+1$, $\forall k\in \mathbb{N}$}{innerhalb\;}{\tcp{\textcolor{gray}{Zähler ist gerade, d.h. $2k$, $\forall k\in \mathbb{N}$}}außerhalb\;}
\caption[Pseudocode Punkt-in-Polygon Algorithmus]{Pseudocode Punkt-in-Polygon Algorithmus.}
\label{fig:pseudocode}
\end{figure}
\end{algorithm}
\DecMargin{1.5em}


Dies ist weiterer Quelltext \ref{fig:punktinpolygon}.
\begin{figure}[H]
\begin{lstlisting}[firstnumber=11]
public boolean enthaeltKoordinate( double lat, double lng ) {

    boolean ungAnzSchnittpkt = false;
    int i, j = polyKanten - 1;

    for( i = 0; i < polyKanten; i++ ) {
        if(( LngArr[ i ] < lng && LngArr[ j ] >= lng ) ||
           ( LngArr[ j ] < lng && LngArr[ i ] >= lng )) {
            if( LatArr[ i ] + (( lng - LngArr[ i ] ) / 
              ( LngArr[ j ] - LngArr[ i ] ) * ( LatArr[ j ] - LatArr[ i ] )) >= lat ) {
                ungAnzSchnittpkt = !ungAnzSchnittpkt;
            }
        }
        j = i;
    }
    return ungAnzSchnittpkt;
} // Ende enthaeltKoordinate()-Methode
\end{lstlisting}
\caption[Auszug der PunktInPolygon-Klasse]{Auszug der \texttt{PunktInPolygon}-Klasse. Dieser zeigt die Implementierung des Punkt-in-Polygon Algorithmus.}
\label{fig:punktinpolygon}
\end{figure} 




\section{Section}



\subsection{Subsection}

\subsubsection{Subsubsection}
